\documentclass[12pt]{article}
\usepackage{pmmeta}
\pmcanonicalname{FunctionalCalculus}
\pmcreated{2013-03-22 17:29:40}
\pmmodified{2013-03-22 17:29:40}
\pmowner{asteroid}{17536}
\pmmodifier{asteroid}{17536}
\pmtitle{functional calculus}
\pmrecord{16}{39882}
\pmprivacy{1}
\pmauthor{asteroid}{17536}
\pmtype{Feature}
\pmcomment{trigger rebuild}
\pmclassification{msc}{47A60}
\pmclassification{msc}{46H30}
\pmrelated{FunctionalCalculusForHermitianMatrices}
\pmrelated{ContinuousFunctionalCalculus2}
\pmrelated{PolynomialFunctionalCalculus}
\pmrelated{BorelFunctionalCalculus}

% this is the default PlanetMath preamble.  as your knowledge
% of TeX increases, you will probably want to edit this, but
% it should be fine as is for beginners.

% almost certainly you want these
\usepackage{amssymb}
\usepackage{amsmath}
\usepackage{amsfonts}

% used for TeXing text within eps files
%\usepackage{psfrag}
% need this for including graphics (\includegraphics)
%\usepackage{graphicx}
% for neatly defining theorems and propositions
%\usepackage{amsthm}
% making logically defined graphics
%%%\usepackage{xypic}

% there are many more packages, add them here as you need them

% define commands here

\begin{document}
\PMlinkescapeword{functional}
\PMlinkescapeword{theory}
\PMlinkescapeword{links}
\PMlinkescapeword{spectrum}

\section{Basic Idea}
Let $X$ be a normed vector space over a field $\mathbb{K}$. Let $T$ be a linear operator in $X$ and $I$ the identity operator in $X$.

The \PMlinkescapetext{term} {\bf functional calculus} refers to a specific process which enables the expression
\begin{displaymath}
f(T)
\end{displaymath}
to make sense as a linear operator in $X$, for certain scalar functions $f: \mathbb{K} \longrightarrow \mathbb{K}$.

At first sight, and for most functions $f$, there is no reason why the above expression should be associated with a particular linear operator.

But, for example, when $f$ is a polynomial $f(x)=a_kx^k + \dots + a_2x^2 + a_1x + a_0$, the expression
\begin{displaymath}
f(T):=a_kT^k + \dots + a_2T^2 + a_1T +a_0I
\end{displaymath}
does indeed refer to a linear operator in $X$.

As another example, when $T$ is a matrix in $\mathbb{R}^n$ or $\mathbb{C}^n$ one is sometimes led to the \PMlinkname{exponential}{MatrixExponential} of $T$
\begin{displaymath}
e^T=\sum_{k=0}^{\infty} \frac{T^k}{k!}
\end{displaymath}
Thus, we are applying the scalar exponential function to a matrix.

Note in this last example that $e^T$ is approximated by polynomials (the partial sums of the series). This provides an idea of how to make sense of $f(T)$ if $f$ can be approximated by polynomials:

If $f$ can be approximated by polynomials $p_n$ then one could try to define
\begin{displaymath}
f(T):=\lim_{n \rightarrow \infty} p_n(T)
\end{displaymath}

But for that one needs to define what ``approximated'' means and to assure the above limit exists.

\section{More abstractly}
There is no reason why one should restrict to linear operators in a normed vector space. In this \PMlinkescapetext{way}, we can consider instead a unital topological algebra $\mathcal{A}$ over a field $\mathbb{K}$.

There is no definition in mathematics of {\bf functional calculus}, but the ideas above show that a \emph{functional calculus for an element} $T \in \mathcal{A}$ should be something like an homomorphism $(\cdot)(T) : \mathcal{F} \longrightarrow \mathcal{A}$ from some topological algebra of scalar functions $\mathcal{F}$ to $\mathcal{A}$, that satisfied the following \PMlinkescapetext{properties}:

\begin{itemize}
\item $\mathcal{F}$ must contain the polynomial functions.
\item $(\cdot)(T)$ is continuous.
\item $p(T)= a_kT^k + \dots + a_2T^2 + a_1T +a_0I$ for each polynomial $p(x)=a_kx^k + \dots + a_2x^2 + a_1x + a_0$, where $I$ denotes the identity element of $\mathcal{A}$.
\end{itemize}

\section{Functional Calculi}

There are some functional calculi of \PMlinkescapetext{interest}. We give a very brief descprition of each one of them (please follows the links for entries with more detailed explanation).

\begin{itemize}
\item \emph{polynomial functional calculus - }

This is valid for any element $T$ in any algebra $\mathcal{A}$. It associates polynomials to elements in the algebra generated by $T$, as discussed above.

\item \emph{\PMlinkescapetext{analytic} functional calculus - }

This is valid for any element $T$ in a complex Banach algebra $\mathcal{A}$. It associates complex analytic functions defined on the \PMlinkname{spectrum}{Spectrum} of $T$ to elements in the algebra generated by $T$.

\item \emph{continuous functional calculus - }

This is valid for normal elements in \PMlinkname{$C^*$-algebras}{CAlgebra}. It associates continuous functions on the spectrum of $T$ to elements in the $C^*$-algebra generated by $T$.

\item \emph{Borel functional calculus - }

This is valid for normal operators $T$ in a von Neumann algebra $\mathcal{A}$. It associates bounded Borel measurable functions on the spectrum of $T$ to elements in the von Neumann algebra generated by $T$.

\item \emph{functional calculus for Hermitian matrices - }

A \PMlinkescapetext{finite dimensional} case. It is valid for Hermitian matrices $T$. It associates real valued functions on the spectrum of $T$ to elements in the algebra generated by $T$.
\end{itemize}

\section{Applications}
\begin{itemize}
\item Functional calculi provide an \PMlinkescapetext{effective way} of constructing new linear operators having specified \PMlinkescapetext{properties} out of given ones.

\item There are strong \PMlinkescapetext{connections} with spectral theory since one usually has $f(\sigma(T)) = \sigma(f(T))$, where $\sigma(\cdot)$ denotes the spectrum of its \PMlinkescapetext{argument}. This is called the spectral mapping theorem.

\item As the \PMlinkescapetext{connections} with spectral theory can possibly show, functional calculi are an \PMlinkescapetext{effective} tool for studying \PMlinkescapetext{operator} equations. For example, they can give sufficient conditions for the existence of a square root $\sqrt{T}$ of an \PMlinkescapetext{operator} $T$.
\end{itemize}
%%%%%
%%%%%
\end{document}
