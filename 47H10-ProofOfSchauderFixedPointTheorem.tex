\documentclass[12pt]{article}
\usepackage{pmmeta}
\pmcanonicalname{ProofOfSchauderFixedPointTheorem}
\pmcreated{2013-03-22 13:45:22}
\pmmodified{2013-03-22 13:45:22}
\pmowner{asteroid}{17536}
\pmmodifier{asteroid}{17536}
\pmtitle{proof of Schauder fixed point theorem}
\pmrecord{10}{34457}
\pmprivacy{1}
\pmauthor{asteroid}{17536}
\pmtype{Proof}
\pmcomment{trigger rebuild}
\pmclassification{msc}{47H10}
\pmclassification{msc}{46T99}
\pmclassification{msc}{46T20}
\pmclassification{msc}{46B50}
\pmclassification{msc}{54H25}

% this is the default PlanetMath preamble.  as your knowledge
% of TeX increases, you will probably want to edit this, but
% it should be fine as is for beginners.

% almost certainly you want these
\usepackage{amssymb}
\usepackage{amsmath}
\usepackage{amsfonts}

% used for TeXing text within eps files
%\usepackage{psfrag}
% need this for including graphics (\includegraphics)
%\usepackage{graphicx}
% for neatly defining theorems and propositions
%\usepackage{amsthm}
% making logically defined graphics
%%%\usepackage{xypic}

% there are many more packages, add them here as you need them

% define commands here
\begin{document}
The idea of the proof is to reduce to the finite dimensional case where we can apply the Brouwer fixed point theorem.

Given $\epsilon>0$ notice that the family of open sets $\{ B_\epsilon(x)\colon x\in K\}$ is an open covering of $K$. Being $K$ compact there exists a finite subcover, i.e. there exists $n$ points $x_1,\ldots, x_n$ of $K$ such that the balls $B_\epsilon(x_i)$ cover the whole set $K$. 

Define the functions $g_1, \ldots, g_n$ by
\begin{displaymath}
g_i(x):=
\begin{cases}
\epsilon -\|x-x_i\|, & \;\;\textrm{if $\|x-x_i\| \leq \epsilon$} \\
0, & \;\;\textrm{if $\|x-x_i\| \geq \epsilon$}
\end{cases}
\end{displaymath}

It is clear that each $g_i$ is continuous, $g_i(x) \geq 0$ and $\sum_{i=1}^n g_i(x) > 0$ for every $x \in K$.

Thus we can define a function in $K$ by
\begin{displaymath}
g(x):= \frac{\sum_{i=1}^n g_i(x) x_i}{\sum_{i=1}^n g_i(x)}
\end{displaymath}

The above function $g$ is a continuous function from $K$ to the convex hull $K_0$ of $x_1, \ldots, x_n$. Moreover one can easily prove the following \PMlinkescapetext{estimate}
\begin{displaymath}
\|g(x)-x\| \leq \epsilon \quad\;\; \forall_{x \in K}
\end{displaymath}

Now, define the function $B:=g \circ f$. The restriction $\tilde{B}$ of $B$ to $K_0$ provides a continuous function $K_0 \longrightarrow K_0$.

Since $K_0$ is compact convex subset of a finite dimensional vector space, we can apply the Brouwer fixed point theorem to assure the existence of $z \in K_0$  such that
\begin{displaymath}
B(z) = \tilde{B}(z) = z
\end{displaymath}

Therefore $g(f(z)) = z$ and we have the inequality
\begin{displaymath}
\|f(z) - z\| = \|f(z) - g(f(z))\| \leq \epsilon
\end{displaymath}

Summarizing, for each $\epsilon >0$ there exists $z = z(\epsilon) \in K$ such that $\|f(z) - z\| \leq \epsilon$. Then

\begin{displaymath}
\forall_{m \in \mathbb{N}} \quad\;\; \exists_{z_m \in K} \quad\;\; \|f(z_m)-z_m\| \leq \frac{1}{m}
\end{displaymath}

As $f(z_m)$ is in the compact space $K$, there is a subsequence $z_{m_k}$ such that $f(z_{m_k}) \longrightarrow x_0$, for some $x_0 \in K$.

We then have
\begin{eqnarray*}
\|z_{m_k} -x_0\| & = & \|z_{m_k} - f(z_{m_k}) +f(z_{m_k}) -x_0\| \\
& \leq & \|f(z_{m_k})-z_{m_k}\| + \|f(z_{m_k}) - x_0\| \\
& \leq & \frac{1}{m_k} + \|f(z_{m_k}) - x_0\| \longrightarrow 0
\end{eqnarray*}
which means that $z_{m_k} \longrightarrow x_0$.

As $f$ is continuous we have $f(z_{m_k}) \longrightarrow f(x_0)$. Both limits of $f(z_{m_k})$ must coincide, so we conclude that
\begin{displaymath}
f(x_0)=x_0
\end{displaymath}

i.e. $f$ has a fixed point. $\square$
%%%%%
%%%%%
\end{document}
