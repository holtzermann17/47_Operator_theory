\documentclass[12pt]{article}
\usepackage{pmmeta}
\pmcanonicalname{TransposeOperator}
\pmcreated{2013-03-22 17:34:19}
\pmmodified{2013-03-22 17:34:19}
\pmowner{asteroid}{17536}
\pmmodifier{asteroid}{17536}
\pmtitle{transpose operator}
\pmrecord{5}{39983}
\pmprivacy{1}
\pmauthor{asteroid}{17536}
\pmtype{Definition}
\pmcomment{trigger rebuild}
\pmclassification{msc}{47A05}
\pmclassification{msc}{46-00}
\pmsynonym{conjugate operator}{TransposeOperator}
\pmrelated{Transpose}
\pmrelated{Adjoint5}

\endmetadata

% this is the default PlanetMath preamble.  as your knowledge
% of TeX increases, you will probably want to edit this, but
% it should be fine as is for beginners.

% almost certainly you want these
\usepackage{amssymb}
\usepackage{amsmath}
\usepackage{amsfonts}

% used for TeXing text within eps files
%\usepackage{psfrag}
% need this for including graphics (\includegraphics)
%\usepackage{graphicx}
% for neatly defining theorems and propositions
%\usepackage{amsthm}
% making logically defined graphics
%%%\usepackage{xypic}

% there are many more packages, add them here as you need them

% define commands here

\begin{document}
Let $X, Y$ be normed vector spaces and $X', Y'$ be their continuous dual spaces.

{\bf \PMlinkescapetext{Definition} -} Let $T:X \longrightarrow Y$ be a bounded linear operator. The operator $T':Y' \longrightarrow X'$ given by
\begin{displaymath}
T'\phi = \phi \circ T , \;\;\; \phi \in Y'
\end{displaymath}
is called the {\bf transpose operator} of $T$ or the {\bf conjugate operator} of $T$.

It is clear that $T'$ is well defined, i.e. $\phi \circ T \in X'$, since the composition of two continuous linear operators is again a continuous linear operator.

Moreover, it can be easily checked that $T'$ is a bounded linear operator.

{\bf Remarks -}
\begin{itemize}
\item When the vector spaces are finite dimensional, the transpose operator corresponds to \PMlinkname{transposing}{Transpose} the matrix associated to it.
\item For Hilbert spaces, a somewhat similar definition is that of adjoint operator. But this two notions do not coincide: while the transpose operator corresponds to the transpose of a matrix, the adjoint operator corresponds to the conjugate transpose of a matrix.
\end{itemize}
%%%%%
%%%%%
\end{document}
