\documentclass[12pt]{article}
\usepackage{pmmeta}
\pmcanonicalname{OperatorNormOfMultiplicationOperatorOnL2}
\pmcreated{2013-04-06 22:14:23}
\pmmodified{2013-04-06 22:14:23}
\pmowner{rspuzio}{6075}
\pmmodifier{rspuzio}{6075}
\pmtitle{operator norm of multiplication operator on $L^2$}
\pmrecord{13}{37798}
\pmprivacy{1}
\pmauthor{rspuzio}{6075}
\pmtype{Theorem}
\pmcomment{trigger rebuild}
\pmclassification{msc}{47B38}

% this is the default PlanetMath preamble.  as your knowledge
% of TeX increases, you will probably want to edit this, but
% it should be fine as is for beginners.

% almost certainly you want these
\usepackage{amssymb}
\usepackage{amsmath}
\usepackage{amsfonts}

% used for TeXing text within eps files
%\usepackage{psfrag}
% need this for including graphics (\includegraphics)
%\usepackage{graphicx}
% for neatly defining theorems and propositions
%\usepackage{amsthm}
% making logically defined graphics
%%%\usepackage{xypic}

% there are many more packages, add them here as you need them

% define commands here
\begin{document}
The operator norm of the multiplication operator $M_\phi$ is the
essential supremum of the absolute value of $\phi$.  (This may be 
expressed as $\| M_\phi \|_\mathrm{op} = \| \phi \|_{L^\infty}$.)
In particular, if $\phi$ is essentially unbounded, the multiplication 
operator is unbounded.

For the time being, assume that $\phi$ is essentially bounded.

On the one hand, the operator norm is bounded by the essential
supremum of the absolute value because, for any $\psi \in L^2$,
\begin{eqnarray*}
\| M_\phi \psi \|_{L^2} &=&  \sqrt{\int \psi(x)^2 \phi(x)^2 \,d\mu(x)} \\
&\le& \sqrt{ (\mathrm{ess } \sup\, \phi^2) \int \psi(x)^2 \,d\mu(x)} \\
&=&  (\mathrm{ess }\sup\, |\phi|)\,\| \psi\|_{L^2} \\
\end{eqnarray*}
and, hence
\[ \| M_\phi \|_\mathrm{op} = \sup {\| M_\phi \psi \|_{L^2} \over \| \psi
\|_{L^2}} \le (\mathrm{ess }\sup\,| \phi|).\]

On the other hand, the operator norm bounds by the essential supremum
of the absolute value .  For any $\epsilon > 0$, the measure of the
set
\[ A = \{ x \mid\,|\phi(x)| \ge \mathrm{ess }\sup\, |\phi| - \epsilon \} \]
is greater than zero.  If $\mu(A) < \infty$, set $B= A$, otherwise let
$B$ be a subset of $A$ whose measure is finite.  Then, if $\chi_B$ is
the characteristic function of $B$, we have
\begin{eqnarray*}
\| M_\phi \chi_B \|_{L^2} &=&  \sqrt{\int \phi(x)^2 \chi_B(x)^2
\,d\mu(x)} \\
&=& \sqrt{\int_B \phi(x)^2 \,d\mu(x)} \\
&\ge& \mu(B) (\mathrm{ess }\sup\, |\phi| - \epsilon)\\
\end{eqnarray*}
and, hence
\[ \| M_\phi \|_\mathrm{op} = \sup {\| M_\phi \psi \|_{L^2} \over \|
\psi \|_{L^2}} \ge {\| M_\phi \chi_B \|_{L^2} \over \| \chi_B
\|_{L^2}} = \mathrm{ess }\sup\, |\phi| - \epsilon.\]
Since this is true for every $\epsilon > 0$, we must have
\[ \| M_\phi \|_\mathrm{op} \ge \mathrm{ess }\sup\, |\phi|.\]
Combining with the inequality in the opposite direction,
\[ \| M_\phi \|_\mathrm{op} = \mathrm{ess }\sup\, |\phi|.\]

It remains to consider the case where $|\phi|$ is essentially
unbounded.  This can be dealt with by a variation on the preceeding
argument.

If $\phi$ is unbounded, then $\mu (\{ x \mid\,|\phi (x)| \ge R\}) > 0$
for all $R>0$.  Furthermore, for any $R>0$, we can find $N>R$ such
that $\mu(A) > 0$, where
\[A = \{ x \mid N+1 \ge |\phi (x)| \ge N\}.\]
 If $\mu(A) < \infty$, set $B= A$, otherwise let
$B$ be a subset of $A$ whose measure is finite.  Then, if $\chi_B$ is
the characteristic function of $B$, we have
\begin{eqnarray*}
\| M_\phi \chi_B \|_{L^2} &=&  \sqrt{\int \phi(x)^2 \chi_B(x)^2
\,d\mu(x)} \\
&=& \sqrt{\int_B \phi(x)^2 \,d\mu(x)} \\
&\ge& \mu(B) N\\
\end{eqnarray*}
and, hence
\[ \| M_\phi \|_\mathrm{op} = \sup {\| M_\phi \psi \|_{L^2} \over \|
\psi \|_{L^2}} \ge {\| M_\phi \chi_B \|_{L^2} \over \| \chi_B
\|_{L^2}} = N \ge R.\]
Since this is true for every $R$, we see that the operator norm is
infinite, i.e. the operator is unbounded.

\end{document}
