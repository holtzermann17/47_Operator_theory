\documentclass[12pt]{article}
\usepackage{pmmeta}
\pmcanonicalname{ProofOfBrouwerFixedPointTheorem1}
\pmcreated{2013-03-22 18:13:24}
\pmmodified{2013-03-22 18:13:24}
\pmowner{uriw}{288}
\pmmodifier{uriw}{288}
\pmtitle{proof of Brouwer fixed point theorem}
\pmrecord{4}{40808}
\pmprivacy{1}
\pmauthor{uriw}{288}
\pmtype{Proof}
\pmcomment{trigger rebuild}
\pmclassification{msc}{47H10}
\pmclassification{msc}{54H25}
\pmclassification{msc}{55M20}

\endmetadata

\usepackage{amsmath, amsthm, amssymb}

\newtheorem{theorem}{Theorem}
\newtheorem{corollary}[theorem]{Corollary}
\newtheorem{lemma}[theorem]{Lemma}
\newtheorem{claim}[theorem]{Claim}
\newtheorem{proposition}[theorem]{Proposition}
\newtheorem{example}[theorem]{Example}
\newtheorem{conjecture}[theorem]{Conjecture}
\newtheorem{remark}[theorem]{Remark}
\newtheorem{definition}[theorem]{Definition}


\begin{document}
The $n$-dimensional simplex $\mathcal{S}_n$ is the following
subset of $\mathbb{R}^{n+1}$
\[
\left\{(\alpha_1,\alpha_2,\ldots,\alpha_{n+1}) \, \Big| \,
\sum_{i=1}^{n+1}\alpha_i=1, \quad \alpha_i\geq0 \quad \forall
i=1,\ldots,n+1\right\}
\]

Given an element $x=\sum_i\alpha_i e_i\in\mathcal{S}_n$ we denote
$[x]_i=\alpha_i$ (i.e., the $i$-th barycentric coordinate). We
also denote $F(x)=\{\, i\, |\, [x]_i \neq 0\}$. An $I$-face of
$\mathcal{S}_n$ is the subset $\{\, x\, |\, F(x)\subseteq I \}$.

As was noted in the statement of the theorem, the 'shape' is
unimportant. Therefore, we will prove the
following variant of the theorem using the KKM lemma.

\begin{theorem}[Brouwer's Fixed Point Theorem]
Let $f:\mathcal{S}_n\to\mathcal{S}_n$ be a continuous function.
Then, $f$ has a fixed point, namely, there is an
$L\in\mathcal{S}_n$ such that $L=f(L)$.
\end{theorem}
\begin{proof}
Clearly, $\sum_{i=1}^n [y]_i = 1$ for any $y\in\mathcal{S}_n$ and
$L=f(L)$ if and only if $[L]_i = [f(L)]_i$ for all
$i=1,2,\ldots,n+1$. For each $i=1,2,\ldots, n+1$ we define the
following subset $C_i$ of $\mathcal{S}_n$:
\[
C_i = \left\{x\in\mathcal{S}_n \,\Big|\, [x]_i\geq[f(x)]_i\right\}
\]
We claim that if $x$ is in some $I$-face of $\mathcal{S}_n$
($I\subseteq\{1,2,\ldots,n+1\}$) then there is an index $i\in I$
such that $x \in C_i$. Indeed, if $x$ is in some $I$-face then
$F(v) \subseteq I$. Thus, if $[x]_i \neq 0$ then $i\in I$. This
shows that
\[
\sum_{i\in I} [x]_i = 1
\]
Assuming by contradiction that $x\not\in C_i$ for all $i\in I$
implies that $[x]_i < [f(x)]_i$ for all $i\in I$. But this leads
to a contradiction as the following inequality shows:
\[
1 = \sum_{i\in I} [x]_i < \sum_{i\in I} [f(x)]_i \leq \sum_{i=1}^n
[f(x)]_i = 1
\]
This dicussion establishes that each $I$-face is contained in the
union $\cup_{i\in I} C_i$. In addition, the subsets $C_i$ are all
closed. Therefore, we have shown that the hypothesis of the KKM
Lemma holds.

By the KKM lemma there is a point $L$ that is in every $C_i$ for
$i=1,2,\ldots,n+1$. We claim that $L$ is a fixed point of $f$.
Indeed, $[L]_i\geq[f(L)]_i\geq0$ for all $i=1,2,\ldots,n+1$ and
thus:
\[
1 = [L]_1 + [L]_2 + \cdots + [L]_{n+1} \geq [f(L)]_1 + [f(L)]_2 +
\cdots + [f(L)]_{n+1} = 1
\]
Therefore, $[L]_i=[f(L)]_i$ for all $i=1,2,\ldots,n+1$ which
implies that $L=f(L)$.
\end{proof}

%%%%%
%%%%%
\end{document}
