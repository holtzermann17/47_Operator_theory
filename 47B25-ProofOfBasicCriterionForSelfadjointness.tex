\documentclass[12pt]{article}
\usepackage{pmmeta}
\pmcanonicalname{ProofOfBasicCriterionForSelfadjointness}
\pmcreated{2013-03-22 14:53:05}
\pmmodified{2013-03-22 14:53:05}
\pmowner{Koro}{127}
\pmmodifier{Koro}{127}
\pmtitle{proof of basic criterion for self-adjointness}
\pmrecord{5}{36564}
\pmprivacy{1}
\pmauthor{Koro}{127}
\pmtype{Proof}
\pmcomment{trigger rebuild}
\pmclassification{msc}{47B25}

\endmetadata

% this is the default PlanetMath preamble.  as your knowledge
% of TeX increases, you will probably want to edit this, but
% it should be fine as is for beginners.

% almost certainly you want these
\usepackage{amssymb}
\usepackage{amsmath}
\usepackage{amsfonts}
\usepackage{mathrsfs}

% used for TeXing text within eps files
%\usepackage{psfrag}
% need this for including graphics (\includegraphics)
%\usepackage{graphicx}
% for neatly defining theorems and propositions
%\usepackage{amsthm}
% making logically defined graphics
%%%\usepackage{xypic}

% there are many more packages, add them here as you need them

% define commands here
\newcommand{\C}{\mathbb{C}}
\newcommand{\R}{\mathbb{R}}
\newcommand{\N}{\mathbb{N}}
\newcommand{\Z}{\mathbb{Z}}
\newcommand{\Per}{\operatorname{Per}}
\newcommand{\Ran}{\operatorname{Ran}}
\newcommand{\Ker}{\operatorname{Ker}}
\begin{document}
\begin{enumerate}
\item $(1\implies 2)$ If $A$ is self-adjoint and $Ax = ix$, then $$i\|x\|^2=(ix,x) = (Ax,x) = (x,A^*x) = (x,Ax) = (x,ix) = \overline{(ix,x)} = -i\|x\|^2,$$
so $x=0$. Similarly we prove that $Ax=-ix$ implies $x=0$. That $A$ is closed follows from the fact that the adjoint of an operator is always closed.
\item $(2\implies 3)$ If $2$ holds, then $\{0\}= \Ker (A^*\pm i)^* = \Ker (A \mp i)^* = \Ran(A\mp i)^\perp$, so that $\Ran{A \mp i}$ is dense in $\mathscr{H}$. Also, since $A$ is symmetric, for $x\in D(A)$,
$$\|(A+i)x\|^2=\|Ax\|^2 + \|x\|^2 + (Ax,ix) + (ix,Ax) = \|Ax\|^2 + \|x\|^2$$
because $(Ax,ix) = (x,iA^*x)= (x,iAx) = -(ix,Ax)$.
Hence $\|x\|\leq \|(A+i)x\|$, so that given a sequence $x_n\in D(A)$ such that $(A+i)x_n \to y$, we have that $\{(A+i)x_n\}$ is a Cauchy sequence and thus $\{x_n\}$ itself is a Cauchy sequence. Hence $\{x_n\}$ converges to some $x\in \mathscr{H}$ and since $A$ is closed it follows that $x\in D(A)$ and $(A+i)x = y$. This proves that $y\in \Ran(A+i)$, so that $\Ran(A+i)$ is closed (and similarly, $\Ran(A-i)$ is closed. Thus $\Ran(A\pm i)=\mathscr{H}$.
\item $(3\implies 1)$ Suppose $3$. If $y\in D(A^*)$, then there is $x\in D(A)$ such that $(A+i)x = (A^*-i)y$. Since $A$ is symmetric, $(A+i)x = (A^*+i)x = (A-i)^*x$, so that $(A^*-i)(x-y)=0$. But since $\Ker(A^*-i) = \Ran(A+i)^\perp = \{0\}$, it follows that $x=y$, so that $y\in D(A)$. Hence $D(A)=D(A^*)$, and therefore $A$ is self-adjoint.
\end{enumerate}
%%%%%
%%%%%
\end{document}
