\documentclass[12pt]{article}
\usepackage{pmmeta}
\pmcanonicalname{LocallyCompactQuantumGroupsFromVonNeumannCAlgebrasWithHaarMeasures}
\pmcreated{2013-03-22 18:24:28}
\pmmodified{2013-03-22 18:24:28}
\pmowner{bci1}{20947}
\pmmodifier{bci1}{20947}
\pmtitle{locally compact quantum groups from von Neumann/$C^*$- algebras with Haar measures}
\pmrecord{16}{41056}
\pmprivacy{1}
\pmauthor{bci1}{20947}
\pmtype{Topic}
\pmcomment{trigger rebuild}
\pmclassification{msc}{47A70}
\pmclassification{msc}{46N50}
\pmclassification{msc}{47L30}
\pmclassification{msc}{47N50}
\pmclassification{msc}{81P15}
\pmclassification{msc}{46C05}
\pmsynonym{locally compact quantum groups}{LocallyCompactQuantumGroupsFromVonNeumannCAlgebrasWithHaarMeasures}
\pmsynonym{quantum groupoids}{LocallyCompactQuantumGroupsFromVonNeumannCAlgebrasWithHaarMeasures}
\pmsynonym{Hopf and weak Hopf algebras}{LocallyCompactQuantumGroupsFromVonNeumannCAlgebrasWithHaarMeasures}
%\pmkeywords{quantum groupoids}
%\pmkeywords{Hopf and weak Hopf algebras}
\pmrelated{HilbertSpace}
\pmrelated{QuantumSpaceTimes}
\pmrelated{VonNeumannAlgebra}
\pmrelated{WeakHopfCAlgebra2}
\pmrelated{ClassificationOfHilbertSpaces}
\pmrelated{QuantumSpaceTimes}
\pmrelated{VonNeumannAlgebra}
\pmrelated{WeakHopfCAlgebra2}
\pmrelated{JordanBanachAndJordanLieAlgebras}
\pmrelated{QuantumLogic}
\pmrelated{Distribution4}
\pmdefines{JBW-algebras}
\pmdefines{$JBW$ algebras}
\pmdefines{$JB$--algebra}
\pmdefines{$JBWA$}
\pmdefines{$JL$}
\pmdefines{Jordan-Banach-von Neumann algebras}
\pmdefines{$CQG_{lc}$}

% this is the default PlanetMath preamble.  as your knowledge
% of TeX increases, you will probably want to edit this, but
% it should be fine as is for beginners.

% almost certainly you want these
\usepackage{amssymb}
\usepackage{amsmath}
\usepackage{amsfonts}

% used for TeXing text within eps files
%\usepackage{psfrag}
% need this for including graphics (\includegraphics)
%\usepackage{graphicx}
% for neatly defining theorems and propositions
%\usepackage{amsthm}
% making logically defined graphics
%%%\usepackage{xypic}

% there are many more packages, add them here as you need them

% define commands here
\usepackage{amsmath, amssymb, amsfonts, amsthm, amscd, latexsym}
%%\usepackage{xypic}
\usepackage[mathscr]{eucal}

\setlength{\textwidth}{6.5in}
%\setlength{\textwidth}{16cm}
\setlength{\textheight}{9.0in}
%\setlength{\textheight}{24cm}

\hoffset=-.75in     %%ps format
%\hoffset=-1.0in     %%hp format
\voffset=-.4in

\theoremstyle{plain}
\newtheorem{lemma}{Lemma}[section]
\newtheorem{proposition}{Proposition}[section]
\newtheorem{theorem}{Theorem}[section]
\newtheorem{corollary}{Corollary}[section]

\theoremstyle{definition}
\newtheorem{definition}{Definition}[section]
\newtheorem{example}{Example}[section]
%\theoremstyle{remark}
\newtheorem{remark}{Remark}[section]
\newtheorem*{notation}{Notation}
\newtheorem*{claim}{Claim}

\renewcommand{\thefootnote}{\ensuremath{\fnsymbol{footnote%%@
}}}
\numberwithin{equation}{section}

\newcommand{\Ad}{{\rm Ad}}
\newcommand{\Aut}{{\rm Aut}}
\newcommand{\Cl}{{\rm Cl}}
\newcommand{\Co}{{\rm Co}}
\newcommand{\DES}{{\rm DES}}
\newcommand{\Diff}{{\rm Diff}}
\newcommand{\Dom}{{\rm Dom}}
\newcommand{\Hol}{{\rm Hol}}
\newcommand{\Mon}{{\rm Mon}}
\newcommand{\Hom}{{\rm Hom}}
\newcommand{\Ker}{{\rm Ker}}
\newcommand{\Ind}{{\rm Ind}}
\newcommand{\IM}{{\rm Im}}
\newcommand{\Is}{{\rm Is}}
\newcommand{\ID}{{\rm id}}
\newcommand{\GL}{{\rm GL}}
\newcommand{\Iso}{{\rm Iso}}
\newcommand{\Sem}{{\rm Sem}}
\newcommand{\St}{{\rm St}}
\newcommand{\Sym}{{\rm Sym}}
\newcommand{\SU}{{\rm SU}}
\newcommand{\Tor}{{\rm Tor}}
\newcommand{\U}{{\rm U}}

\newcommand{\A}{\mathcal A}
\newcommand{\Ce}{\mathcal C}
\newcommand{\D}{\mathcal D}
\newcommand{\E}{\mathcal E}
\newcommand{\F}{\mathcal F}
\newcommand{\G}{\mathcal G}
\newcommand{\Q}{\mathcal Q}
\newcommand{\R}{\mathcal R}
\newcommand{\cS}{\mathcal S}
\newcommand{\cU}{\mathcal U}
\newcommand{\W}{\mathcal W}

\newcommand{\bA}{\mathbb{A}}
\newcommand{\bB}{\mathbb{B}}
\newcommand{\bC}{\mathbb{C}}
\newcommand{\bD}{\mathbb{D}}
\newcommand{\bE}{\mathbb{E}}
\newcommand{\bF}{\mathbb{F}}
\newcommand{\bG}{\mathbb{G}}
\newcommand{\bK}{\mathbb{K}}
\newcommand{\bM}{\mathbb{M}}
\newcommand{\bN}{\mathbb{N}}
\newcommand{\bO}{\mathbb{O}}
\newcommand{\bP}{\mathbb{P}}
\newcommand{\bR}{\mathbb{R}}
\newcommand{\bV}{\mathbb{V}}
\newcommand{\bZ}{\mathbb{Z}}

\newcommand{\bfE}{\mathbf{E}}
\newcommand{\bfX}{\mathbf{X}}
\newcommand{\bfY}{\mathbf{Y}}
\newcommand{\bfZ}{\mathbf{Z}}

\renewcommand{\O}{\Omega}
\renewcommand{\o}{\omega}
\newcommand{\vp}{\varphi}
\newcommand{\vep}{\varepsilon}

\newcommand{\diag}{{\rm diag}}
\newcommand{\grp}{{\mathbb G}}
\newcommand{\dgrp}{{\mathbb D}}
\newcommand{\desp}{{\mathbb D^{\rm{es}}}}
\newcommand{\Geod}{{\rm Geod}}
\newcommand{\geod}{{\rm geod}}
\newcommand{\hgr}{{\mathbb H}}
\newcommand{\mgr}{{\mathbb M}}
\newcommand{\ob}{{\rm Ob}}
\newcommand{\obg}{{\rm Ob(\mathbb G)}}
\newcommand{\obgp}{{\rm Ob(\mathbb G')}}
\newcommand{\obh}{{\rm Ob(\mathbb H)}}
\newcommand{\Osmooth}{{\Omega^{\infty}(X,*)}}
\newcommand{\ghomotop}{{\rho_2^{\square}}}
\newcommand{\gcalp}{{\mathbb G(\mathcal P)}}

\newcommand{\rf}{{R_{\mathcal F}}}
\newcommand{\glob}{{\rm glob}}
\newcommand{\loc}{{\rm loc}}
\newcommand{\TOP}{{\rm TOP}}

\newcommand{\wti}{\widetilde}
\newcommand{\what}{\widehat}

\renewcommand{\a}{\alpha}
\newcommand{\be}{\beta}
\newcommand{\ga}{\gamma}
\newcommand{\Ga}{\Gamma}
\newcommand{\de}{\delta}
\newcommand{\del}{\partial}
\newcommand{\ka}{\kappa}
\newcommand{\si}{\sigma}
\newcommand{\ta}{\tau}
\newcommand{\med}{\medbreak}
\newcommand{\medn}{\medbreak \noindent}
\newcommand{\bign}{\bigbreak \noindent}
\newcommand{\lra}{{\longrightarrow}}
\newcommand{\ra}{{\rightarrow}}
\newcommand{\rat}{{\rightarrowtail}}
\newcommand{\oset}[1]{\overset {#1}{\ra}}
\newcommand{\osetl}[1]{\overset {#1}{\lra}}
\newcommand{\hr}{{\hookrightarrow}}

\begin{document}
\subsection{Hilbert spaces, Von Neumann algebras and Quantum Groups}
 John von Neumann introduced a mathematical foundation for Quantum Mechanics in the form of  
\PMlinkname{$W^*$-algebras}{WeakHopfCAlgebra2} 
of (quantum) bounded operators in a (quantum:= presumed \emph{separable}, i.e. with a countable basis) Hilbert space $H_S$. Recently, such 
\PMlinkname{von Neumann algebras, $W^*$}{WeakHopfCAlgebra2} and/or (more generally) C*-algebras are, for example, employed to define 
\PMlinkname{locally compact quantum groups $CQG_{lc}$}{LocallyCompactQuantumGroup} by equipping such 
\PMlinkname{algebras with a co-associative multiplication}{WeakHopfCAlgebra2}
and also with associated, both left-- and right-- Haar measures, defined by two semi-finite normal weights 
\cite{Vainerman2003}.

\subsubsection{Remark on Jordan-Banach-von Neumann (JBW) algebras, $JBWA$}
A \emph{Jordan--Banach algebra} (a JB--algebra for short) is both a real Jordan algebra and a
Banach space, where for all $S, T \in \mathfrak A_{\bR}$, we have the following.

A \emph{JLB--algebra} is a $JB$--algebra $\mathfrak A_{\bR}$ together with a Poisson bracket for 
which it becomes a Jordan--Lie algebra $JL$ for some $\hslash^2 \geq 0$~. Such JLB--algebras often 
constitute the real part of several widely studied complex associative algebras. 
For the purpose of quantization, there are fundamental relations between 
\PMlinkname{$\mathfrak A^{sa}$, JLB and Poisson algebras}{JordanBanachAndJordanLieAlgebras}.
\bigbreak
\begin{definition}
A JB--algebra which is monotone complete and admits a separating set of normal sets is
called a \emph{JBW-algebra}.
\end{definition}

These appeared in the work of von Neumann who developed an \emph{orthomodular lattice theory of projections on $\mathcal L(H)$} on which to study \emph{quantum logic}. BW-algebras have the following property: whereas $\mathfrak A^{sa}$ is a J(L)B--algebra, the self-adjoint part of a von Neumann algebra is a JBW--algebra.



\begin{thebibliography}{9}

\bibitem{Vainerman2003}
Leonid Vainerman. 2003.
\PMlinkexternal{``Locally Compact Quantum Groups and Groupoids'': \\
Proceedings of the Meeting of Theoretical Physicists and Mathematicians}{http://planetmath.org/?op=getobj&from=books&id=160}, Strasbourg, February 21-23, 2002., Walter de Gruyter Gmbh \& Co: Berlin.

\bibitem{QTF}
Von Neumann and the 
\PMlinkexternal{Foundations of Quantum Theory.}{http://plato.stanford.edu/entries/qt-nvd/}

\bibitem{Bohm66}
B$\"o$hm, A., 1966, Rigged Hilbert Space and Mathematical Description of Physical Systems, {\em Physica A}, 236: 485-549. 

\bibitem{Bohm89}
B$\"o$hm, A. and Gadella, M., 1989, \emph{Dirac Kets, Gamow Vectors and Gel'fand Triplets}, New York: Springer-Verlag. 

\bibitem{DJ81}
Dixmier, J., 1981, \emph{Von Neumann Algebras}, Amsterdam: North-Holland Publishing Company. [First published in French in 1957: \emph{Les Alge'bres d'Ope'rateurs dans l'Espace Hilbertien}, Paris: Gauthier-Villars.] 

\bibitem{GINM43}
Gelfand, I. and Neumark, M., 1943, On the Imbedding of Normed Rings into the Ring of Operators in Hilbert Space, 
{\em Recueil Mathe'matique} [Matematicheskii Sbornik] Nouvelle Se'rie, 12 [54]: 197-213. [Reprinted in C*-algebras: 1943-1993, in the series Contemporary Mathematics, 167, Providence, R.I. : American Mathematical Society, 1994.] 

\bibitem{Alex55}
Grothendieck, A., 1955, Produits Tensoriels Topologiques et Espaces Nucl$\'e$aires, 
\emph{Memoirs of the American Mathematical Society}, 16: 1-140. 

\bibitem{HS90}
Horuzhy, S. S., 1990, {\em Introduction to Algebraic Quantum Field Theory}, Dordrecht: Kluwer Academic Publishers. 

\bibitem{JV55}
J. von Neumann.,1955, {\em Mathematical Foundations of Quantum Mechanics.}, Princeton, NJ: Princeton University Press. [First published in German in 1932: {\em Mathematische Grundlagen der Quantenmechanik}, Berlin: Springer.]

\bibitem{JV37}
J. von Neumann, 1937, {\em Quantum Mechanics of Infinite Systems}, first published in (Radei and Statzner 2001, 249-268). [A mimeographed version of a lecture given at Pauli's seminar held at the Institute for Advanced Study in 1937, John von Neumann Archive, Library of Congress, Washington, D.C.] 


\end{thebibliography}
 

%%%%%
%%%%%
\end{document}
