\documentclass[12pt]{article}
\usepackage{pmmeta}
\pmcanonicalname{BasicCriterionForSelfadjointness}
\pmcreated{2013-03-22 14:53:02}
\pmmodified{2013-03-22 14:53:02}
\pmowner{Koro}{127}
\pmmodifier{Koro}{127}
\pmtitle{basic criterion for self-adjointness}
\pmrecord{5}{36563}
\pmprivacy{1}
\pmauthor{Koro}{127}
\pmtype{Theorem}
\pmcomment{trigger rebuild}
\pmclassification{msc}{47B25}

\endmetadata

% this is the default PlanetMath preamble.  as your knowledge
% of TeX increases, you will probably want to edit this, but
% it should be fine as is for beginners.

% almost certainly you want these
\usepackage{amssymb}
\usepackage{amsmath}
\usepackage{amsfonts}
\usepackage{mathrsfs}

% used for TeXing text within eps files
%\usepackage{psfrag}
% need this for including graphics (\includegraphics)
%\usepackage{graphicx}
% for neatly defining theorems and propositions
%\usepackage{amsthm}
% making logically defined graphics
%%%\usepackage{xypic}

% there are many more packages, add them here as you need them

% define commands here
\newcommand{\C}{\mathbb{C}}
\newcommand{\R}{\mathbb{R}}
\newcommand{\N}{\mathbb{N}}
\newcommand{\Z}{\mathbb{Z}}
\newcommand{\Per}{\operatorname{Per}}
\begin{document}
Let $A\colon D(A)\subset \mathscr{H}\to \mathscr{H}$ be a symmetric operator on a Hilbert space. The following are equivalent:
\begin{enumerate}
\item $A=A^*$ (i.e $A$ is self-adjoint);
\item $\operatorname{Ker}(A^* \pm i) = \{0\}$ and $A$ is closed;
\item $\operatorname{Ran}(A \pm i) = \mathscr{H}$.
\end{enumerate}

\emph{Remark:} $A+\lambda$ represents the operator $A+\lambda I\colon D(A)\subset \mathscr{H}\to \mathscr{H}$, and $\operatorname{Ker}$ and $\operatorname{Ran}$ stand for kernel and range, respectively.

A similar version for essential self-adjointness is an easy corollary of the above. The following are equivalent:

\begin{enumerate}
\item $\overline A=A^*$ (i.e. $A$ is essentially self-adjoint);
\item $\operatorname{Ker}(A^*\pm i) = \{0\}$;
\item $\operatorname{Ran}(A \pm i)$ is dense in $\mathscr{H}$.
\end{enumerate}
%%%%%
%%%%%
\end{document}
