\documentclass[12pt]{article}
\usepackage{pmmeta}
\pmcanonicalname{OperatorInducedByAMeasurePreservingMap}
\pmcreated{2013-03-22 17:59:19}
\pmmodified{2013-03-22 17:59:19}
\pmowner{asteroid}{17536}
\pmmodifier{asteroid}{17536}
\pmtitle{operator induced by a measure preserving map}
\pmrecord{7}{40499}
\pmprivacy{1}
\pmauthor{asteroid}{17536}
\pmtype{Definition}
\pmcomment{trigger rebuild}
\pmclassification{msc}{47A35}
\pmclassification{msc}{28D05}
\pmclassification{msc}{37A05}
\pmdefines{isometry induced by a measure preserving map}

\endmetadata

% this is the default PlanetMath preamble.  as your knowledge
% of TeX increases, you will probably want to edit this, but
% it should be fine as is for beginners.

% almost certainly you want these
\usepackage{amssymb}
\usepackage{amsmath}
\usepackage{amsfonts}

% used for TeXing text within eps files
%\usepackage{psfrag}
% need this for including graphics (\includegraphics)
%\usepackage{graphicx}
% for neatly defining theorems and propositions
%\usepackage{amsthm}
% making logically defined graphics
%%%\usepackage{xypic}

% there are many more packages, add them here as you need them

% define commands here

\begin{document}
\PMlinkescapephrase{theorem}
\PMlinkescapephrase{induced}
\PMlinkescapephrase{property}
\PMlinkescapephrase{properties}
\PMlinkescapephrase{side}

\section{Induced Operators}

Let $(X_1, \mathfrak{B}_1, \mu_1)$ and $(X_2, \mathfrak{B}_2, \mu_2)$ be measure spaces and denote by $L^0(X_1)$ and $L^0(X_2)$ the corresponding spaces of measurable functions (with values in $\mathbb{C}$).

{\bf Definition -} If $T: X_1 \longrightarrow X_2$ is a measure-preserving transformation we can define an operator $U_T:L^0(X_2) \longrightarrow L^0(X_1)$ by

\begin{displaymath}
(U_T f)(x):=f(Tx)\,, \qquad\qquad f \in L^0(X_2),\; x \in X_1
\end{displaymath}

The operator $U_T$ is called the \emph{\PMlinkescapetext{operator induced} by $T$}.

 Many ideas in ergodic theory can be explored by studying this operator.

\section{Basic Properties}

The following \PMlinkescapetext{properties} are clear:

\begin{itemize}
\item $U_T$ is linear.
\item $U_T$ maps real valued functions to real valued functions.
\item If $f \geq 0$ then $U_Tf \geq 0$ 
\item $U_T k = k$ for every constant function $k$.
\item $U_T(fg)=U_T(f)U_T(g)$.
\item $U_T$ maps characteristic functions to characteristic functions. Moreover, $U_T \chi_B = \chi_{T^{-1}B}$, for every measurable set $B \in \mathfrak{B}_2$.
\item If $T_1:X_1 \longrightarrow X_2$ and $T_2:X_2 \longrightarrow X_3$ are measure preserving maps, then $U_{T_2 \circ T_1} = U_{T_1}U_{T_2}$.
\end{itemize}

\section{Preserving Integrals}


{\bf Theorem 1 -} If $f \in L^0(X_2)$ then $\int_{X_1} U_Tf\;d\mu_1 = \int_{X_2} f\;d\mu_2$, where if one side does not exist or is infinite, then the other side has the same property.


\section{Induced Isometries}
 It can further be seen that a measure-preserving transformation induces an isometry between \PMlinkname{$L^p$-spaces}{LpSpace}, for $p \geq 1$.


{\bf Theorem 2 -} Let $p \geq 1$. We have that $U_T(L^p(X_2)) \subseteq L^p(X_1)$ and, moreover,
\begin{displaymath}
\|U_T(f)\|_p = \|f\|_p\,, \qquad\qquad \text{for all}\; f \in L^p(X_2)
\end{displaymath}

$\,$

Thus, when restricted to $L^p$-spaces, $U_T$ is called the \emph{isometry induced by $T$}.
%%%%%
%%%%%
\end{document}
