\documentclass[12pt]{article}
\usepackage{pmmeta}
\pmcanonicalname{ClassicalIsoperimetricProblem}
\pmcreated{2013-03-22 19:10:26}
\pmmodified{2013-03-22 19:10:26}
\pmowner{pahio}{2872}
\pmmodifier{pahio}{2872}
\pmtitle{classical isoperimetric problem}
\pmrecord{22}{42081}
\pmprivacy{1}
\pmauthor{pahio}{2872}
\pmtype{Example}
\pmcomment{trigger rebuild}
\pmclassification{msc}{47A60}
\pmclassification{msc}{49K22}
\pmclassification{msc}{49K05}
\pmrelated{LagrangeMultiplierMethod}
\pmrelated{CircularSegment}
\pmrelated{AngleOfViewOfALineSegment}

% this is the default PlanetMath preamble.  as your knowledge
% of TeX increases, you will probably want to edit this, but
% it should be fine as is for beginners.

% almost certainly you want these
\usepackage{amssymb}
\usepackage{amsmath}
\usepackage{amsfonts}

% used for TeXing text within eps files
%\usepackage{psfrag}
% need this for including graphics (\includegraphics)
%\usepackage{graphicx}
% for neatly defining theorems and propositions
 \usepackage{amsthm}
% making logically defined graphics
%%%\usepackage{xypic}
\usepackage{pstricks}
\usepackage{pst-plot}

% there are many more packages, add them here as you need them

% define commands here

\theoremstyle{definition}
\newtheorem*{thmplain}{Theorem}

\begin{document}
\PMlinkescapeword{variation}
The points $a$ and $b$ on the $x$-axis have to be \PMlinkescapetext{connected} by an arc with a given \PMlinkname{length}{ArcLength} $l$ such that the area between the $x$-axis and the arc is as great as possible.\\

Denote the equation of the searched arc by\, $y = y(x)$.\, The task, which belongs to the \PMlinkname{isoperimetric problems}{IsoperimetricProblem}, can be formulated as
\begin{align}
\mbox{to maximise} \quad \int_a^b\!y\,dx
\end{align}
under the constraint condition
\begin{align}
\int_a^b\!\sqrt{1\!+\!y'^{\,2}}\,dx \;=\; l.
\end{align}
We have the integrands
$$f(x,\,y,\,y') \;\equiv\; y, \quad  g(x,\,y,\,y') \;\equiv\; \sqrt{1\!+\!y'^{\,2}}.$$
The \PMlinkescapetext{conditional} variation problem for the functional in (1) may be considered as a free variation problem (without conditions) for the functional $\int_a^b(f\!-\!\lambda g)\,dx$ where $\lambda$ is a Lagrange multiplier.\, For this end we need the \PMlinkname{Euler--Lagrange differential equation}{EulerLagrangeDifferentialEquation}
\begin{align}
\frac{\partial}{\partial y}(f\!-\!\lambda g)-\frac{d}{dx}\frac{\partial}{\partial y'}(f\!-\!\lambda g) \;=\; 0.
\end{align}
Since the expression $f\!-\!\lambda g$ does not depend explicitly on $x$, the differential equation (3) has, by the Beltrami identity, a first integral of the form
$$(f\!-\!\lambda g)-y'\!\cdot\!(f'_{y'}\!-\!\lambda g'_{y'}) \;=\; C_2,$$
which reads simply
$$y-\frac{\lambda}{\sqrt{1\!+\!y'^{\,2}}} \;=\; C_2.$$
This differential equation may be written
$$y' \;\equiv\; \frac{dy}{dx} \;=\; \frac{\sqrt{\lambda^2\!-\!(y\!-\!C_2)^2}}{y\!-\!C_2},$$
where one can \PMlinkname{separate the variables}{SeparationOfVariables} and integrate, obtaining the equation
$$(x\!-\!C_1)^2+(y\!-\!C_2)^2 \;=\; \lambda^2$$
of a circle.\, Here, the parametres $C_1,\,C_2,\,\lambda$ may be determined from the conditions 
$$y(a) \;=\; y(b) \;=\; 0, \quad \mbox{arc length} \;=\; l.$$
Thus the extremal of this variational problem is  a \PMlinkname{circular arc}{CircularSegment} connecting the given points.

Note that in every point of the arc, the angle of view of the line segment between the given points is constant.


\begin{center}
\begin{pspicture}(-4,-3)(4,2)
\psdots[linewidth=0.02](-1.5,0)(1.5,0)
\psline{->}(-3,0)(3,0)
\psarc[linecolor=red](0,-2.6){3}{60}{120}
\rput(3.1,-0.25){$x$}
\rput(-1.5,-0.3){$a$}
\rput(+1.5,-0.3){$b$}
\rput(0,0.7){$l$}
\rput(-4,-2){.}
\rput(+4,+2){.}
\end{pspicture}
\end{center}

%%%%%
%%%%%
\end{document}
