\documentclass[12pt]{article}
\usepackage{pmmeta}
\pmcanonicalname{BrouwerFixedPointInOneDimension}
\pmcreated{2013-03-22 13:46:25}
\pmmodified{2013-03-22 13:46:25}
\pmowner{mathcam}{2727}
\pmmodifier{mathcam}{2727}
\pmtitle{Brouwer fixed point in one dimension}
\pmrecord{10}{34480}
\pmprivacy{1}
\pmauthor{mathcam}{2727}
\pmtype{Proof}
\pmcomment{trigger rebuild}
\pmclassification{msc}{47H10}
\pmclassification{msc}{54H25}
\pmclassification{msc}{55M20}
\pmrelated{LipschitzCondition}

\endmetadata

% this is the default PlanetMath preamble.  as your knowledge
% of TeX increases, you will probably want to edit this, but
% it should be fine as is for beginners.

% almost certainly you want these
\usepackage{amssymb}
\usepackage{amsmath}
\usepackage{amsfonts}

\newcommand{\sR}[0]{\mathbb{R}}
\newcommand{\sC}[0]{\mathbb{C}}
\newcommand{\sN}[0]{\mathbb{N}}
\newcommand{\sZ}[0]{\mathbb{Z}}
\begin{document}
\PMlinkescapeword{satisfies}
{\bf Theorem 1} \cite{mukherjea, adams}
Suppose $f$ is a continuous function
$f:[-1,1]\to [-1,1]$. Then $f$ has a fixed point, i.e.,
there is a $x$ such that $f(x)=x$.

{\bf Proof} (Following \cite{mukherjea})
We can assume that $f(-1)>-1$ and $f(+1)<1$, since otherwise
there is nothing to prove. Then, consider the function $g:[-1,1]\to \sR$
defined by $g(x)=f(x)-x$. It satisfies
\begin{eqnarray*}
g(+1) &<& 0,\\
g(-1) &>& 0,
\end{eqnarray*}
so by the intermediate value theorem, there is a point $x$
such that $g(x)=0$, i.e., $f(x)=x$. $\Box$

Assuming slightly more of the function $f$ yields the
Banach fixed point theorem. In one dimension it states the following:

{\bf Theorem 2} Suppose $f:[-1,1]\to [-1,1]$ is a function that satisfies the 
following condition:
\begin{itemize}
\item[] for some constant $C\in[0,1)$, we have for each $a,b\in[-1,1]$, 
$$|f(b)-f(a)|\le C |b-a|.$$
\end{itemize}
Then $f$ has  a unique fixed point in $[-1,1]$. In other words, there exists
one and only one point $x\in[-1,1]$ such that $f(x)=x$. 

{\bf Remarks}
The fixed point in Theorem 2 can be found by iteration from any $s\in[-1,1]$ as follows: 
first choose some $s\in[-1,1]$. 
Then form $s_1=f(s)$, then $s_2=f(s_1)$, and generally $s_n=f(s_{n-1})$. 
As $n\to \infty$, $s_n$ approaches the fixed point for $f$. More details 
are given on the entry for the Banach fixed point theorem. 
A function that satisfies the
 condition in Theorem 2 is called a contraction mapping. Such mappings also satisfy the
\PMlinkname{Lipschitz condition}{LipschitzCondition}.

\begin{thebibliography}{9}
 \bibitem{mukherjea}
 A. Mukherjea, K. Pothoven,
 \emph{Real and Functional analysis},
 Plenum press, 1978.
 \end{thebibliography}
%%%%%
%%%%%
\end{document}
