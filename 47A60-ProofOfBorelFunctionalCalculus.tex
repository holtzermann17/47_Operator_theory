\documentclass[12pt]{article}
\usepackage{pmmeta}
\pmcanonicalname{ProofOfBorelFunctionalCalculus}
\pmcreated{2013-03-22 18:50:26}
\pmmodified{2013-03-22 18:50:26}
\pmowner{asteroid}{17536}
\pmmodifier{asteroid}{17536}
\pmtitle{proof of Borel functional calculus}
\pmrecord{8}{41646}
\pmprivacy{1}
\pmauthor{asteroid}{17536}
\pmtype{Proof}
\pmcomment{trigger rebuild}
\pmclassification{msc}{47A60}
\pmclassification{msc}{46L10}
\pmclassification{msc}{46H30}

\endmetadata

% this is the default PlanetMath preamble.  as your knowledge
% of TeX increases, you will probably want to edit this, but
% it should be fine as is for beginners.

% almost certainly you want these
\usepackage{amssymb}
\usepackage{amsmath}
\usepackage{amsfonts}

% used for TeXing text within eps files
%\usepackage{psfrag}
% need this for including graphics (\includegraphics)
%\usepackage{graphicx}
% for neatly defining theorems and propositions
%\usepackage{amsthm}
% making logically defined graphics
%%%\usepackage{xypic}

% there are many more packages, add them here as you need them

% define commands here

\begin{document}
\PMlinkescapeword{property}
\PMlinkescapeword{properties}
\PMlinkescapeword{section}
\PMlinkescapeword{multiplicative}
\PMlinkescapeword{parent}

In this entry we give a proof of the main result about the Borel functional calculus (\PMlinkescapetext{Theorem} 1 on the parent entry). We will restate here the result for convenience. Please, check the parent entry for the details on notation. 

{\bf Theorem -} \emph{Let $T$ be a normal operator in $B(H)$ and $\pi:C(\sigma(T)) \longrightarrow B(H)$ the unital *-homomorphism corresponding to the continuous functional calculus for $T$. Then, $\pi$ extends uniquely to a *-homomorphism $\widetilde{\pi}: B(\sigma(T)) \longrightarrow B(H)$ that is continuous from the $\mu$-topology to the weak operator topology. Moreover, each operator $\pi(f)$ lies in \PMlinkname{strong operator}{OperatorTopologies} closure of the unital *-algebra generated by $T$.}

$\,$

{\bf \emph{Proof :}} First we shall prove the existence of the \PMlinkname{extension}{ExtensionOfAFunction} $\widetilde{\pi}$ with the described continuity property, then we shall prove its uniqueness and for last we shall prove the last assertion of the theorem about the image of $\widetilde{\pi}$. For simplicity the proofs of some auxiliary results are given at the end of the entry

\subsubsection{Existence}

For each pair of vectors $\xi, \eta \in H$ consider the linear functional $\phi_{\xi, \eta} : C(\sigma(T)) \to \mathbb{C}$ given by
\begin{align*}
\phi_{\xi, \eta}(f) := \langle \pi(f) \xi, \eta \rangle
\end{align*}
This linear functional is bounded with norm at most $\| \xi\| \|\eta \|$ because
\begin{align*}
\|\phi_{\xi, \eta}(f)\| \leq \|\pi\|\,\|f\|\, \|\xi\| \|\eta \| = \|f\|\, \|\xi\| \|\eta \|
\end{align*}
where the last equality comes from the fact that $\pi$ is a *-isomorphism between \PMlinkname{$C^*$-algebras}{CAlgebras}, therefore having norm 1 (see \PMlinkname{this entry}{HomomorphismsOfCAlgebrasAreContinuous}).

By the \PMlinkname{Riesz representation theorem}{RieszRepresentationTheoremOfLinearFunctionalsOnFunctionSpaces} there is a unique complex Radon measure $\mu_{\xi, \eta}$ in $\sigma(T)$ such that
\begin{align}
\langle \pi(f) \xi, \eta \rangle = \int_{\sigma(T)} f \,d\mu_{\xi, \eta}\,, \qquad\qquad \forall f \in C(\sigma(T))
\end{align}

Consider now the mapping $(\xi, \eta) \mapsto \mu_{\xi, \eta}$. This mapping has the following properties, whoose proofs are given at the end of the entry (section Auxiliary Results):
\begin{itemize}
\item[] {\bf a)} $\mu_{\lambda_1\xi_1+ \lambda_2\xi_2, \eta} = \lambda_1\mu_{\xi_1,\eta} + \lambda_2\mu_{\xi_2, \eta}$, for all $\lambda_1, \lambda_2 \in \mathbb{C}$.
\item[] {\bf b)} $\mu_{\xi, \lambda_1 \eta_1+ \lambda_2 \eta_2} = \overline{\lambda_1}\mu_{\xi,\eta_1} + \overline{\lambda_2}\mu_{\xi, \eta_2}$, for all $\lambda_1, \lambda_2 \in \mathbb{C}$.
\item[] {\bf c)} $\mu_{\xi, \eta} = \overline{\mu_{\eta, \xi}}$.
\item[] {\bf d)} $g \cdot \mu_{\xi, \eta} = \mu_{\pi(g) \xi, \eta}$, for all $g \in C(\sigma(T)$.
\end{itemize}

Therefore, for each function $f \in B(\sigma(T))$ we have a sesquilinear form in $H$ given by

\begin{align*}
[\xi, \eta] := \int_{\sigma(T)} f\, d\mu_{\xi, \eta}
\end{align*}

Moreover, this sesquilinear form is \PMlinkname{bounded}{BoundedSesquilinearForm} with norm at most $\|f\| \, \|\xi\| \|\eta\|$. By the Riesz lemma on bounded sesquilinear forms, there is a unique operator $\widetilde{\pi}(f) \in B(H)$ such that

\begin{align*}
\langle \widetilde{\pi}(f)\xi, \eta \rangle := \int_{\sigma(T)} f\, d\mu_{\xi, \eta}\,, \qquad\qquad \xi, \eta \in H
\end{align*}

We will now see that the mapping $\widetilde{\pi}: B(\sigma(T) \to B(H)$, such that $f \mapsto \widetilde{\pi}(f)$, has the desired properties stated in the theorem.

First, it is clear that $\widetilde{\pi}$ is linear. Also clear is the fact that $\widetilde{\pi}$ coincides with $\pi$ in $C(\sigma(T))$, because of equality (1) and the uniqueness part of Riesz lemma on sesquilinear forms. Now, for any real valued function $f \in B(\sigma(T))$ we have that

\begin{eqnarray*}
\langle \widetilde{\pi}(f)^*\xi, \eta \rangle & = & \overline{\langle \widetilde{\pi}(f) \eta , \xi \rangle} \\
& = & \overline{\int_{\sigma(T)} f\, d\mu_{\eta, \xi}}\\
& = & \int_{\sigma(T)} f\, d\mu_{\xi, \eta}\\
& = & \langle \pi(f) \xi, \eta \rangle
\end{eqnarray*}
which means that $\widetilde{\pi}(f)^* = \widetilde{\pi}(f)$, i.e. $\widetilde{\pi}(f)$ is self-adjoint. Decomposing an arbitrary function $f \in B(\sigma(T))$ in its real and imaginary parts we see that $\widetilde{\pi}(f)^*=\widetilde{\pi}(\overline{f})$.

We now show that $\widetilde{\pi}$ is multiplicative, i.e. $\widetilde{\pi}(fg)=\widetilde{\pi}(f)\widetilde{\pi}(g)$ for all $f, g \in B(\sigma(T))$. For that we need an additional property of the measures $\mu_{\xi, \eta}$, whoose proof is also at the end of the entry:
\begin{itemize}
\item[] {\bf e)} $f\cdot \mu_{\xi, \eta} = \mu_{\xi, \widetilde{\pi}(f)^* \eta}$, for all $f \in B(\sigma(T))$.
\end{itemize}

Given $f, g \in B(\sigma(T))$ we have, for every $\xi, \eta \in H$,
\begin{eqnarray*}
\langle \widetilde{\pi}(fg) \xi, \eta \rangle & = & \int_{\sigma(T)} fg\, d\mu_{\xi, \eta} = \int_{\sigma_(T)} g\, d\mu_{\xi, \widetilde{\pi}(f)* \eta}\\
& = & \langle \widetilde{\pi}(g) \xi, \widetilde{\pi}(f)^* \eta \rangle = \langle \widetilde{\pi}(f) \widetilde{\pi}(g) \xi, \eta \rangle
\end{eqnarray*}
and therefore $\widetilde{\pi}(fg)= \widetilde{\pi}(f)\widetilde{\pi}(g)$. Thus, $\widetilde{\pi}: B(\sigma(T)) \to B(H)$ is a *-homomorphism that extends $\pi$.

\subsubsection{Continuity Property}
We now prove that the above defined $\widetilde{\pi}$ is continuous from the $\mu$-topology to the weak operator topology.

Let $\{f_i\}$ be a net of functions in $B(\sigma(T))$ that converge in the $\mu$-topology to a function $f \in B(\sigma(T))$. This means that for all Radon measures $\nu$ in $\sigma(T)$ we have $\int f_i \, d\nu \to \int f\, d\nu$.

Now for all $\xi, \eta \in H$ we have
\begin{align*}
|\langle \widetilde{\pi}(f_i - f) \xi, \eta \rangle| = \big| \int_{\sigma(T)} f_i-f \, d\mu_{\xi, \eta} \big| \to 0
\end{align*}
Hence, $\widetilde{\pi}(f_i)$ converges to $\widetilde{\pi}(f)$ in the weak operator topology.

\subsubsection{Uniqueness}

Let $\pi':B(\sigma(T)) \to B(H)$ be another *-homomorphism that extends $\pi$ and is continuous from the $\mu$-topology to the weak operator topology. For any measurable subset $S \subset \sigma(T)$ consider the set
\begin{align*}
W:=\{(U,K): U \supset S \text{is open and} K \subset S \text{is compact}\}
\end{align*}
We give this set the partial order $\leq$ such that $(U_1,K_1) \leq (U_2, K_2)$ whenever $U_2 \subset U_1$ and $K_1 \subset K_2$. For any pair $(U,K) \in W$ there is a continuous function $f_{U, K} \in C(\sigma(T))$ such that $f$ takes values on the interval $[0,1]$, $f|_K = 1$ and $\mathrm{supp} f \subset U$ (see \PMlinkname{this entry}{ApplicationsOfUrysohnsLemmaToLocallyCompactHausdorffSpaces}).

We claim that $f_{U, K}$ converges to $\chi_S$ in the $\mu$-topology. In fact, given a complex Radon measure $\nu$ in $\sigma(T)$, there is for every $\epsilon > 0$ a pair $(U_0, K_0) \in W$ such that $|\nu|(U_0 \setminus K_0) < \epsilon$. Of course, for all pairs $(U, K)$ such that $(U_0, K_0) \leq (U, K)$ we also have $|\nu|(U \setminus K) < \epsilon$. Hence, we have
\begin{align*}
\big| \int_{\sigma(T)} f_{U, K} - \chi_S \,d\nu \big| \leq \int_{U \setminus K} |f_{U, K} - \chi_S| \,d|\nu| \leq \epsilon
\end{align*}
We conclude that $f_{U,K}$ converges to $\chi_S$ in the $\mu$-topology.

Since $\pi'$ is continuous from the $\mu$-topology to the weak operator topology we must have
\begin{align*}
\langle \pi'(f_{U,V})\xi, \eta \rangle \longrightarrow \langle \pi'(\chi_S) \xi, \eta \rangle
\end{align*}
But since $\pi'$ and $\widetilde{\pi}$ coincide with $\pi$ on $C(\sigma(T))$ we also have
\begin{align*}
\langle \pi'(f_{U,V})\xi, \eta \rangle = \langle \widetilde{\pi}(f_{U,V})\xi, \eta \rangle  \longrightarrow \langle \widetilde{\pi}(\chi_S) \xi, \eta \rangle
\end{align*}
Hence, for any characteristic function $\chi_S$ we have $\pi'(\chi_S) = \widetilde{\pi}(\chi_S)$. Since any function $f \in B(\sigma(T))$ can be uniformly approximated by simple functions it follows that $\pi'(f) = \widetilde{\pi}(f)$, and we have proved the uniqueness of $\widetilde{\pi}$.

\subsubsection{Image of $\widetilde{\pi}$}
Let $\mathcal{A}$ be the unital *-algebra generated by $T$. We now prove that for any $f \in B(\sigma(T))$, the operator $\widetilde{\pi}(f)$ lies in the strong operator closure of $\mathcal{A}$, i.e. lies in the von Neumann algebra generated by $T$. For that it is enough to prove that $\widetilde{\pi}(f)$ is in the double commutant $\mathcal{A}''$ of $\mathcal{A}$.

Recall from the continuous functional calculus that $\pi(f)$ is in the norm closure of $\mathcal{A}$, and hence in $\mathcal{A}''$, for every $f \in C(\sigma(T))$.

We have seen above that for each characteristic function $\chi_S$ there is a net $f_{U, K}$ of functions in $C(\sigma(T))$ such that $\widetilde{\pi}(f_{U,K}) \to \widetilde{\pi}(\chi_S)$ in the weak operator topology. Given an element $R$ in the commutant of $\mathcal{A}$ we have
\begin{align*}
\langle \widetilde{\pi}(f_{U,K}) R\; \xi, \eta \rangle = \langle R\, \widetilde{\pi}(f_{U,K})\; \xi, \eta \rangle\,, \qquad\qquad \forall \xi, \eta \in H
\end{align*}
The first term converges to $\langle \widetilde{\pi}(\chi_S) R\; \xi, \eta \rangle$, whereas the second to $\langle R\,\widetilde{\pi}(\chi_S)\; \xi, \eta \rangle$. Thus, $\widetilde{\pi}(\chi_S) R = R\,\widetilde{\pi}(\chi_S)$, and therefore $\widetilde{\pi}(\chi_S) \in \mathcal{A}''$.

Since every function $f \in B(\sigma(T))$ can be uniformly approximated by simple functions, it follows that $\widetilde{\pi}(f) \in \mathcal{A}''$.

\subsubsection{Auxiliary Results}

In this section we prove the properties of the measures $\mu_{\xi, \eta}$ stated and used above.

\begin{itemize}
\item[] {\bf a)} For all functions $f \in C(\sigma(T))$ we have $\langle \pi(f) (\lambda_1\xi_1 + \lambda_2 \xi_2), \eta \rangle = \lambda_1\langle \pi(f) \xi_1 , \eta \rangle + \lambda_2 \langle \pi(f) \xi_2, \eta \rangle$. Hence,
\begin{align*}
\int_{\sigma(T)} f \, d\mu_{\lambda_1\xi_1+\lambda_2\xi_2, \eta} = \lambda_1\int_{\sigma(T)} f \, d\mu_{\xi_1, \eta} + \lambda_2\int_{\sigma(T)} f \,d\mu_{\xi_2, \eta}
\end{align*}
Since this holds for every $f \in C(\sigma(T))$, the uniqueness part of \PMlinkescapetext{Riesz representation theorem} tells us that
\begin{align*}
\mu_{\lambda_1\xi_1+\lambda_2\xi_2, \eta} = \lambda_1\mu_{\xi_1, \eta} + \lambda_2\mu_{\xi_2, \eta}
\end{align*}
\item[] {\bf b)} The proof is similar to a).
\item[] {\bf c)} For every $f \in C(\sigma(T))$ we have
\begin{eqnarray*}
\int_{\sigma(T)} f \, d\mu_{\xi, \eta} & = & \langle \pi(f) \xi, \eta \rangle = \overline{\langle \pi(\overline{f}) \eta, \xi \rangle}\\
& = & \overline{\int_{\sigma(T)} \overline{f} \, d\mu_{\eta, \xi}} = \int_{\sigma(T)} f \, d\overline{\mu_{\eta, \xi}}
\end{eqnarray*}
Hence we conclude that $\mu_{xi, \eta} = \overline{\mu_{\eta, \xi}}$.
\item[] {\bf d)} For every $f \in C(\sigma(T))$ we have
\begin{eqnarray*}
\int_{\sigma(T)} f \, dg \cdot\mu_{\xi, \eta} & = & \int_{\sigma(T)} fg \, d\mu_{\xi, \eta} = \langle \pi(fg) \xi, \eta \rangle\\
& = & \langle \pi(f)\pi(g) \xi, \eta \rangle = \int_{\sigma(T)} f \, d\mu_{\pi(g)\xi, \eta}
\end{eqnarray*}
Hence, $g \cdot \mu_{\xi, \eta} = \mu_{\pi(g)\xi, \eta}$.
\item[] {\bf e)} For all $h \in C(\sigma(T))$ we have
\begin{eqnarray*}
\int_{\sigma(T)} h \, df \cdot\mu_{\xi, \eta} & = & \int_{\sigma(T)} hf \, d\mu_{\xi, \eta} = \int_{\sigma(T)} f \, dh \cdot\mu_{\xi, \eta}\\
& = & \int_{\sigma(T)} f \, d\mu_{\pi(f)\xi, \eta} = \langle \widetilde{\pi}(f)\pi(g) \xi, \eta \rangle\\
& = & \langle \pi(h) \xi, \widetilde{\pi}(f)^*\eta \rangle = \int_{\sigma(T)} h \, d\mu_{\xi, \widetilde{\pi}(f)^*\eta}
\end{eqnarray*}
Hence, $f \cdot \mu_{\xi, \eta} = \mu_{\xi, \widetilde{\pi}(f)^* \eta}$.
\end{itemize}
%%%%%
%%%%%
\end{document}
