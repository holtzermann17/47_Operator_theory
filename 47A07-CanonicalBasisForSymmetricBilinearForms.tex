\documentclass[12pt]{article}
\usepackage{pmmeta}
\pmcanonicalname{CanonicalBasisForSymmetricBilinearForms}
\pmcreated{2013-03-22 14:56:25}
\pmmodified{2013-03-22 14:56:25}
\pmowner{Mathprof}{13753}
\pmmodifier{Mathprof}{13753}
\pmtitle{canonical basis for symmetric bilinear forms}
\pmrecord{7}{36631}
\pmprivacy{1}
\pmauthor{Mathprof}{13753}
\pmtype{Definition}
\pmcomment{trigger rebuild}
\pmclassification{msc}{47A07}
\pmclassification{msc}{11E39}
\pmclassification{msc}{15A63}
\pmdefines{Sylvester's Law of Inertia}

% this is the default PlanetMath preamble.  as your knowledge
% of TeX increases, you will probably want to edit this, but
% it should be fine as is for beginners.

% almost certainly you want these
\usepackage{amssymb}
\usepackage{amsmath}
\usepackage{amsfonts}

% used for TeXing text within eps files
%\usepackage{psfrag}
% need this for including graphics (\includegraphics)
%\usepackage{graphicx}
% for neatly defining theorems and propositions
%\usepackage{amsthm}
% making logically defined graphics
%%%\usepackage{xypic}

% there are many more packages, add them here as you need them

% define commands here
\begin{document}
If $B:V \times V \rightarrow K$ is a symmetric bilinear form 
over a finite-dimensional vector space, where the characteristic of the field is
not 2, 
then we may prove that there is an orthogonal basis such that $B$ is represented by 

$$
\bordermatrix{& \cr
& a_{1} &  0  & \ldots & 0\cr
& 0  &  a_{2} & \ldots & 0\cr
& \vdots & \vdots & \ddots & \vdots\cr
& 0  &   0       &\ldots & a_{n}\cr
}
$$

Recall that a bilinear form has a well-defined rank, and denote this by $r$.

If $K = \mathbb{R}$ we may choose a basis such that $a_1 = \cdots = a_t = 1$, 
$a_{t+1} = \cdots = a_{t+p} = -1$ and $a_{t+p+j} = 0$, for some integers $p$ and $t$,
where $1 \le j \le n-t-p$.
Furthermore, these integers are \emph{invariants} of the bilinear form. 
This is known as \emph{Sylvester's Law of Inertia}. 
$B$ is \emph{positive definite} if and only if
 $t = n$, $p = 0$. Such a form constitutes a \emph{real inner product space}.

If $K = \mathbb{C}$ we may go further and choose a basis such that $a_1 = \cdots = a_r = 1$ and 
$a_{r + j} = 0$, where $1 \le j \le n-r$.

If $K = F_p$ we may choose a basis such that $a_1 = \cdots = a_{r-1} = 1$, 
 
$a_r = n$ or $a_r = 1$; 
and $a_{r+j} = 0$, where $1 \le j \le n-r$, and 
 $n$ is the least positive quadratic non-residue.
%%%%%
%%%%%
\end{document}
