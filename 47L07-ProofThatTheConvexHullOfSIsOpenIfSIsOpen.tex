\documentclass[12pt]{article}
\usepackage{pmmeta}
\pmcanonicalname{ProofThatTheConvexHullOfSIsOpenIfSIsOpen}
\pmcreated{2013-03-22 14:09:48}
\pmmodified{2013-03-22 14:09:48}
\pmowner{archibal}{4430}
\pmmodifier{archibal}{4430}
\pmtitle{proof that the convex hull of $S$ is open if $S$ is open}
\pmrecord{6}{35587}
\pmprivacy{1}
\pmauthor{archibal}{4430}
\pmtype{Proof}
\pmcomment{trigger rebuild}
\pmclassification{msc}{47L07}
\pmclassification{msc}{46A55}

\endmetadata

% this is the default PlanetMath preamble.  as your knowledge
% of TeX increases, you will probably want to edit this, but
% it should be fine as is for beginners.

% almost certainly you want these
\usepackage{amssymb}
\usepackage{amsmath}
\usepackage{amsfonts}

% used for TeXing text within eps files
%\usepackage{psfrag}
% need this for including graphics (\includegraphics)
%\usepackage{graphicx}
% for neatly defining theorems and propositions
%\usepackage{amsthm}
% making logically defined graphics
%%%\usepackage{xypic}

% there are many more packages, add them here as you need them

% define commands here
\begin{document}
Let $S$ be an open set in some topological vector space $V$.  For any sequence of positive real numbers $\Lambda = (\lambda_1,\ldots,\lambda_n)$ with $\sum_{i=1}^n \lambda_i = 1$ define 
\[
S_\Lambda = \left\{x\in V \text{ such that } x=\sum_{i=1}^n \lambda_i s_i \text{ for }s_i\in S\right\}.
\]
Then since addition and scalar multiplication are both open maps, each $S_\Lambda$ is open.  Finally, the convex hull is clearly just 
\[
\bigcup_\Lambda S_\Lambda,
\]
which is therefore open.
%%%%%
%%%%%
\end{document}
