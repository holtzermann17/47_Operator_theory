\documentclass[12pt]{article}
\usepackage{pmmeta}
\pmcanonicalname{ClosedOperator}
\pmcreated{2013-03-22 13:48:20}
\pmmodified{2013-03-22 13:48:20}
\pmowner{Koro}{127}
\pmmodifier{Koro}{127}
\pmtitle{closed operator}
\pmrecord{9}{34526}
\pmprivacy{1}
\pmauthor{Koro}{127}
\pmtype{Definition}
\pmcomment{trigger rebuild}
\pmclassification{msc}{47A05}
\pmsynonym{closed}{ClosedOperator}
\pmdefines{closure}
\pmdefines{closable}
\pmdefines{core}

% this is the default PlanetMath preamble.  as your knowledge
% of TeX increases, you will probably want to edit this, but
% it should be fine as is for beginners.

% almost certainly you want these
\usepackage{amssymb}
\usepackage{amsmath}
\usepackage{amsfonts}
\usepackage{mathrsfs}

% used for TeXing text within eps files
%\usepackage{psfrag}
% need this for including graphics (\includegraphics)
%\usepackage{graphicx}
% for neatly defining theorems and propositions
%\usepackage{amsthm}
% making logically defined graphics
%%%\usepackage{xypic}

% there are many more packages, add them here as you need them

% define commands here
\newcommand{\C}{\mathbb{C}}
\newcommand{\R}{\mathbb{R}}
\newcommand{\N}{\mathbb{N}}
\newcommand{\Z}{\mathbb{Z}}
\newcommand{\Per}{\operatorname{Per}}
\begin{document}
Let $B$ be a Banach space. 
A linear operator $A\colon\mathscr{D}(A)\subset B\to B$ is said to be \PMlinkescapetext{\textbf{closed}} if
for every sequence $\{x_n\}_{n\in \N}$ in $\mathscr{D}(A)$ converging to $x\in B$ such that $Ax_n\xrightarrow[n\to\infty]{} y\in B$, it holds $x\in\mathscr{D}(A)$ and $Ax = y$. 
Equivalently, $A$ is closed if its graph is closed in $B\oplus B$.

Given an operator $A$, not necessarily closed, if the closure of its graph in $B\oplus B$ happens to be the graph of some operator, we call that operator the \textbf{closure} of $A$, and we say that $A$ is \textbf{closable}. We denote the closure of $A$ by $\overline{A}$. It follows easily that $A$ is the restriction of $\overline{A}$ to $\mathscr{D}(A)$. 

A \textbf{core} of a closable operator is a subset $\mathscr{C}$ of $\mathscr{D}(A)$ such that the closure of the restriction of $A$ to $\mathscr{C}$ is $\overline{A}$.

The following properties are easily checked:
\begin{enumerate}
\item Any bounded linear operator defined on the whole space $B$ is closed;
\item If $A$ is closed then $A-\lambda I$ is closed;
\item If $A$ is closed and it has an inverse, then $A^{-1}$ is also closed;
\item An operator $A$ admits a closure if and only if for every pair of sequences $\{x_n\}$ and $\{y_n\}$ in $\mathscr{D}(A)$, both converging to $z\in B$, and such that both $\{Ax_n\}$ and $\{Ay_n\}$ converge, it holds $\lim_n Ax_n = \lim_n Ay_n$.
\end{enumerate}
%%%%%
%%%%%
\end{document}
