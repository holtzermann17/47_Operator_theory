\documentclass[12pt]{article}
\usepackage{pmmeta}
\pmcanonicalname{PolarDecompositionInVonNeumannAlgebras}
\pmcreated{2013-03-22 17:28:54}
\pmmodified{2013-03-22 17:28:54}
\pmowner{asteroid}{17536}
\pmmodifier{asteroid}{17536}
\pmtitle{polar decomposition in von Neumann algebras}
\pmrecord{7}{39868}
\pmprivacy{1}
\pmauthor{asteroid}{17536}
\pmtype{Result}
\pmcomment{trigger rebuild}
\pmclassification{msc}{47A05}
\pmclassification{msc}{46L10}

\endmetadata

% this is the default PlanetMath preamble.  as your knowledge
% of TeX increases, you will probably want to edit this, but
% it should be fine as is for beginners.

% almost certainly you want these
\usepackage{amssymb}
\usepackage{amsmath}
\usepackage{amsfonts}

% used for TeXing text within eps files
%\usepackage{psfrag}
% need this for including graphics (\includegraphics)
%\usepackage{graphicx}
% for neatly defining theorems and propositions
%\usepackage{amsthm}
% making logically defined graphics
%%%\usepackage{xypic}

% there are many more packages, add them here as you need them

% define commands here

\begin{document}
{\bf \PMlinkescapetext{Proposition} -} Let $\mathcal{M}$ be a von Neumann algebra acting on a Hilbert space $H$ and $T \in \mathcal{M}$. If $T = VR$ is the polar decomposition for $T$ with $Ker V = Ker R$, then both $V$ and $R$ belong to $\mathcal{M}$.

{\bf Proof :}
\begin{itemize}
\item As $\mathcal{M}$ is a \PMlinkname{$C^*$-algebra}{CAlgebra}, it is known that $R = \sqrt{T^*T}$ belongs to $\mathcal{M}$. (proof will be added later)

\item To see that $V$ also belongs to $\mathcal{M}$, by the double commutant theorem, it suffices to show that $V$ belongs to $\mathcal{M}''$ (the double commutant of $\mathcal{M}$).

Suppose $S \in \mathcal{M}'$. We intend to prove that $V$ commutes with $S$. 

For $x \in H$ we have that
\begin{displaymath}
TSx=STx=SVRx
\end{displaymath}
and
\begin{displaymath}
TSx=VRSx=VSRx
\end{displaymath}

So $SV$ and $VS$ agree on $\overline{Ran\; R}$.

As $R$ is self-adjoint, $\overline{Ran\; R}^{\perp}=KerR$, and so it remains to show that $SV$ and $VS$ agree on $Ker R$. Recall that, by hypothesis, $Ker R = Ker V$.

Let $x \in Ker R$. We have that $RSx=SRx=0$ and therefore
\begin{displaymath}
S(Ker R) \subseteq Ker R = Ker V
\end{displaymath}
and so we can conclude that $VS$ is identically zero in $Ker R$.

Clearly $SV$ is also identically zero on $Ker R = Ker V$.

Thus $VS$ and $SV$ agree on $Ker R$. Therefore $SV=VS$ and so $V \in \mathcal{M}''=\mathcal{M}\; \square$

\end{itemize}
%%%%%
%%%%%
\end{document}
